\documentclass[UTF8]{ctexart}
\usepackage{graphicx}
\usepackage{geometry}
\usepackage{fancyhdr}
\geometry{papersize={21.0cm,29.7cm}}
\geometry{left=3.18cm,right=2.54cm,top=2.54cm,bottom=3.18cm}
\title{目标模型描述文档}
\author{吴康康}
\date{2015年10月15日}
\pagestyle{fancy}
\lhead{}
\chead{目标模型文档}
\rhead{大学生锻炼助手}
\lfoot{}
\cfoot{\thepage}
\rfoot{}
\renewcommand{\headrulewidth}{0.4pt}
\renewcommand{\headwidth}{\textwidth}
\renewcommand{\footrulewidth}{0pt}

\begin{document}
\maketitle
\tableofcontents

\section{前言}

\subsection{小组成员}

\begin{center}
\begin{tabular}{|c|c|c|}
\hline
成员名&学号&联系方式\\
\hline
吴康康&131250085&padeoe@gmail.com\\
\hline
殷迪&131250021&yd13@software.nju.edu.cn\\
\hline
黄威&131250080&hw13@software.nju.edu.cn\\
\hline
朱方圆&131250114&zfy13@software.nju.edu.cn\\
\hline
\end{tabular}
\end{center}

\subsection{更新日志}

\begin{center}
\begin{tabular}{|c|c|}
\hline
更新日期&更新内容\\
\hline
2015-10-15&初始文档\\
\hline
\end{tabular}
\end{center}

\subsection{文档描述}
  本文使用面向目标的需求工程方法,建立了“大学生锻炼助手”的目标模型。

\section{高层次目标的获取}


\section{目标精化}
\subsection{获取对高层次目标的描述}
\subsection{从高层次目标描述中发现AND精化关系}
\subsection{从高层次目标描述中发现OR精化关系}
\subsection{考虑阻碍目标实现的情况}
\subsection{发现目标冲突关系}
\subsection{完善层次结构}

\section{目标实现}
\subsection{将底层目标分配给主体}
\subsection{设计最底层目标的操作(任务)}


\end{document}
